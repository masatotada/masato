% !TEX root=../main.tex
\chapter{序論}
\label{chap:introduction}%labelはmain.texとラベル付けされている

\section{はじめに}
皮膚がんはがんの中では発生頻度の低いが,非常に死亡率が高いがんである.皮膚がんには様々な種類があるが,その中でもメラノーマと呼ばれる皮膚がんは発生率が急激に増加しており,皮膚がんによる死亡者の75\%を占めている.実際,アメリカでは2016年に約76000人もの人がメラノーマと診断され,約10000人もの死亡者が出ている\cite{skin1}.また,メラノーマの推定5年生存率は初期で発見された場合は99\%以上の生存率があるが,末期に発見された場合は約14\%似まで生存率が低下してしまう.そのため,早期発見をし,早期治療をすることが重要となってくる\cite{skin2}.\\
 メラノーマの診断は,ダーモスコピーで得た画像を分析して行う.ダーモスコピーは,色素性皮膚病変の構造を詳細に調べることのできる診断方法である\cite{skin1}.診断医はダーモスコピーで得た画像からABCDルールと呼ばれる医師の経験などに依存する,手作業のみの診断によって行われている\cite{skin3}.この方法ではメラノーマの発生初期のものを発見することは非常に困難であり,医師によっての診断にもばらつきが出てしまう.そこで,深層学習\\
 近年,インターネットの普及や計算機の性能の向上によって医療画像解析を含む様々のビジョンタスクにおいて深層学習の利用は急速に発展しており,その有効性を検証してきた.本研究では,幾つのComputer Visionタスクにおいて精度向上を成功した畳み込み演算とself-Attention 構造を統合したボトルネックTransformer モデル(BoTNet)\cite{botnet}を利用し,高精度な皮膚病変識別モデルの構築を目指す.特に,バックボーンCNNで抽出した高レベルの特徴マップでは,入力画像を持つ空間的な情報を高度的に集約されたが,多様なチャネル方向の特徴を得られていることから,従来の空間的self-Attentionの取り組むだけではなく,特徴のチャネル方向の長距離依存性をモデリングできるself-Attention機構を構築する.具体的に,チャネル方向のSelf-attentionを用いて特徴間の相互関係を取り入れてから,空間方向の特徴間の関係を捉えるdepth-wise畳み込み層演算を利用した転置Transformer モデルを提案する.提案手法の有効性を示すために,HAM10000 のデータセットを用いて実験を行う.
\section{論文構成}
本論文の構成は以下の通り.第2章では,本研究の関連研究について紹介する.第3章では,提案手法のモデルの全体構造やtransformer,feed forward networkについて紹介する.第4章では,実験について説明し,その結果を示す.第5章では,結論を述べる.